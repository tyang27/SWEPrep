\begin{comment}
\subsubsection{Java}
\begin{itemize}
    \itemsep0em
    \item Lambdas:
    \begin{lstlisting}[language=java]
    (T1 t1, ...) -> { t1...};
    \end{lstlisting}
    \item Streams:
    \begin{lstlisting}[language=java]
    collection.stream()
        .map(lambda) //1:1
        .flatMap(lambda) //1:*
        .forEach()
        .sorted()
        .distinct()
        .filter(lambda).
        .concat(stream)
        .limit(int)
        .toArray()/.collect(Collectors.toList());
    \end{lstlisting}
\end{itemize}

\subsubsection{REST APIs}
\begin{itemize}
    \itemsep0em
    \item No saved state e.g. with HTTP
    \item Everything is in the query which is good for scalability and indepenent growth of client server
    \item GET, POST, PUT, DELETE
\end{itemize}
\end{comment}





\begin{comment}
    \item Singleton - only one instance created, global access point.
\begin{lstlisting}[language=java]
class Singleton {
    private static Singleton instance = new Singleton();
    private Singleton() {}
    public static Singleton getInstance() { return instance; }
}
// Usage: Singleton.getInstance();
\end{lstlisting}
    \item Factory - produce items that implement the same interface.
\begin{lstlisting}[language=java]
interface FactoryItem { }
class Item1 implements FactoryItem { }
class Factory {
    static FactoryItem buildItem(String type) {
        if (type.equalsIgnoreCase("Item1")) { return new Item1(); }
        return null;
    }
}
// Usage: FactoryItem item = Factory.buildItem(Item1");
\end{lstlisting}
    \item Builder - method chaining makes it easier to set multiple fields.
\begin{lstlisting}[language=java]
public class Item {
    private final int field1;
    private Item() {}
    Item(Builder builder) { field1 = builder.field1; }
    static class Builder {
        private int field1;
        private Builder();
        static Builder newInstance() { return new Builder(); }
        Builder setField1(int field1) {
            this.field1 = field1;
            return this;
        }
        Item build() { return new Item(this); }
    }
}
// Usage: Item.Builder.newInstance().setField1(1)./*...*/.build();
\end{lstlisting}

    \item Prototype
\begin{lstlisting}[language=java]
abstract class Item implements Cloneable {
    protected String itemName;
    abstract void addItem();
    public Object clone() {
        try {
            return super.clone();
        } catch (CloneNotSupporteException e) {
            e.printStackTrace();
        }
        return null;
    }
}
class Item1 extends Item {/*...*/}
class ItemStore {
    Map<String, Item> protoMap = new HashMap<>();
    static { protoMap.put("Item1", new Item1()); }
    public static Prototype getItem(String item) {
        return (Item) protoMap.get(item).clone();
    }
}
\end{lstlisting}
    \item Adapter
\begin{lstlisting}[language=java]
interface Male { abstract void beingDumb(); }
interface Female { abstract void beingSmart(); }
class male1 implements Male {}
class female1 implements Female {}
class Adapter {
    Male m;
    Adapter(Male male) { this.m = m; }
    public void beingSmart() { m.beingDumb(); }
}
// Usage: Adapter(new Male()).beingSmart(); --> new Female().beingSmart();
\end{lstlisting}
    \item State - uses polymorphism to change the behavior of the object.
\begin{lstlisting}[language=java]
interface State { public void fx(Context ctxt); }
class State1 implements State {/*...*/}
class Context {
    private State currentState;
    Context() { currentState = new State1(); }
    void setState(State state) { currentState = state; }
    void fx() { currentState.fx(this); }
}
// Usage: Context c = new Context(); c.setState(new State2()); c.fx();
\end{lstlisting}
    %\item Strategy
    \item Observer
\begin{lstlisting}[language=java]
class Display {
    private int field1;
    void update(int field1) { field1 = field1; display(); }
    void display() {/*...*/}
}
class Data {
    Display display;
    private int field1;
    Data(Display display) { display = display; }
    void updateField1() {/*...*/}
    void dataChanged() { display.update(field1); }
}
\end{lstlisting}
    \item Visitor
\begin{lstlisting}[language=java]
interface Visitor { void visit(Boyfriend bf); void visit(Bestfriend f); }
interface Friend { void sleepover(Visitor visitor); }
class Boyfriend implements Friend {/*...*/}
class Bestfriend implements Friend {/*...*/}
class Me implements Visitor {/*...*/}
// Usage: Make an collection of Friends, and Me
//  for (Friend f : friends) friend.sleepover(me);
\end{lstlisting}
\end{comment}