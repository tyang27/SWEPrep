\documentclass[10pt]{article}
\usepackage[utf8]{inputenc}
\usepackage{verbatim}
\usepackage[margin=0.5in]{geometry}
\usepackage{amsmath}
\usepackage{amsfonts}
\usepackage{listings}
\usepackage{listings}
\usepackage{xcolor}

\lstset{
    tabsize=2,
    showstringspaces=false,
    keywordstyle=\color{blue},
    basicstyle=\ttfamily,
    commentstyle=\color{gray}\ttfamily
}


\title{Software Interview Prep}
\author{tyang27}
\date{August 2019}

\begin{document}

%\maketitle

\section{Topics}


\subsection{Sorting and searching}

\subsubsection{Binary search - $O(\log n)$}
\textbf{Recursive}:
\begin{lstlisting}[language=java]
int search(int[] A, int key, int low, int high) {
    if (high < low) {
        // return low; // ceiling
        // return high; // floor
        return -1;
    }
    int mid = low + (high-low)/2;
    if (A[mid] == key) return mid;
    else if (A[mid] > key) return search(A, key, low, mid-1);
    else return search(A, key, mid+1, high);
}
\end{lstlisting}
\textbf{Iterative}:
\begin{lstlisting}[language=java]
int search(int[] A, int key) {
    int low = 0; int high = A.length-1;
    while (low <= high) {
        int mid = low + (high - low) / 2;
        if (A[mid] == key) return mid;
        if (key < A[mid]) high = mid-1;
        if (key > A[mid]) low = mid+1;
    }
    return -1;
}
\end{lstlisting}{}

\subsubsection{Merge sort - $O(n\log n)$}
Partition into smaller and smaller arrays, then merge to sort.
\begin{lstlisting}[language=java]
void sort(int[] A) {
    if (A.length <= 1) return;
    
    int mid = A.length/2;
    int[] left = new int[mid];
    int[] right = new int[A.length-mid];
    
    System.arraycopy(A, 0, left, 0, mid);
    System.arraycopy(A, mid, right, 0, A.length-mid);
    sort(left); sort(right); merge(A, left, right);
}
void merge(int[] A, int[] left, int[] right) {
    int l, r, i; l = r = i = 0;
    while (l < left.length || r < right.length) {
        if (l < left.length && r < r.length)
            A[i++] = compare(left[l], right[r]) ? right[r++] : left[l++];
        else if (l < left.length)
            A[i++] = left[l++];
        else if (r < right.length)
            A[i++] = right[r++];
    }    
}
\end{lstlisting}

\subsubsection{Quick sort - $O(n^2)$ worst case, $O(n\log n)$ average case}
%Partition into higher and lower group, then recursively sort.
\begin{lstlisting}[language=java]
void sort(int[] A, int low, int high) {
    if (high < low) return;
    int pivot = partition(A, low, high);
    sort(A, low, pivot-1); sort(A, pivot+1, high);
}
int partition(int[] A, int low, int high) {
    int insPt = low-1;
    for (int curr = low; curr < high; curr++)
        if (A[curr] <= A[high]) swap(A, ++insPt, curr);
    swap(A, ++insPt, high);
    return insPt;
}
\end{lstlisting}

\begin{comment}
\subsubsection{Count sort - $O(n+r)$}
Count frequencies of each element, then unpack.
\begin{lstlisting}[language=java]
void sort(int[] A) {
    if (A.length == 0) return;
    int min = Arrays.stream(A).min().getAsInt();
    int max = Arrays.stream(A).max().getAsInt();
    int[] counts = new int[max-min+1];
    for (int i : A) counts[i-min]++;
    
    int j = 0;
    for (int i=0; i < A.length; i++) {
        while (counts[j] == 0) j++;
        A[i] = min + j;
        counts[j]--;
    }
}
\end{lstlisting}
\end{comment}


\subsubsection{Radix sort - $O(nk/d)$}
\begin{lstlisting}[language=java]
void sort(int[] A) {                                  
    for (int i = 0; i < MAX_LEN; i++) {
        List<List<Integer>> B = new ArrayList<>(BUCKETS);                  
        for (int b = 0; b < BUCKETS; b++) B.add(new ArrayList<>());
        for (int a : A) B.get(placeValue(a, i)).add(a);
        for (int b = 0, unpacker = 0; b < BUCKETS; k++)
            for (int a : B.get(b))
                A[unpacker++] = a;                           
    }
}
\end{lstlisting}

\subsubsection{Median/order stats}
\textbf{Median of medians - $O(n)$ worst and best case}\\
\textbf{Quick select - $O(n^2)$ worst case but $O(n)$ average}\\
Partition, then recurse based on the size of partition created by pivot.
\begin{lstlisting}[language=java]
public static int search(int[] A, int k, int low, int high) {
    if (high == low) return A[low];
    int pivot = partition(A, low, high);
    if (k < pivot) return search(A, k, low, pivot-1);
    else if (k > pivot) return search(A, k, pivot+1, high);
    else return A[pivot];
}
\end{lstlisting}







\begin{comment}
\subsubsection{Insertion sort - $O(n^2)$}
Preserves loop invariant by inserting next element right of cursor into invariant.
\begin{itemize}
    \itemsep0em 
    \item Adaptive - good for mostly sorted data sets.
    \item Stable - preserves order of equal items.
    \item In-place - no auxillary data structures.
    \item Online - does not need whole input to start sorting.
    \item Insertion finding step can be sped up with binary search.
\end{itemize}
\begin{lstlisting}[language=java]
void sort(int[] A) {
    for (int cursor=1; cursor < A.length; cursor++) {
        int cursorVal = A[cursor];
        int left = cursor - 1;
        while (left >= 0 && A[left] > cursorVal) {
            A[left+1] = A[left];
            left--;
        }
        A[left+1] = cursorVal;
    }
}
\end{lstlisting}

\subsubsection{Selection sort - $O(n^2)$}
Preserves loop invariant by putting the next largest element right of cursor as next element of invariant.   
\begin{lstlisting}[language=java]
sort(int[] A) {
    for (int cursor=0; cursor < A.length-1; cursor++) {
        int minIndex = cursor;
        for (int right = cursor+1; right < A.length; right++) {
            if (compare(A[minIndex], A[right])
                minIndex = right;
        }
        if (minIndex != cursor)
            swap(A, cursor, minIndex);
    }
}
\end{lstlisting}

\subsubsection{Bubble sort}
Not very good or interesting...
\end{comment}

\subsubsection{Permutations}
\textbf{All - O(n!)}
% Head recursion to the end so the last few numbers will be a permutation. Tail recursion to reset it to the last state.
\begin{lstlisting}[language=java]
void permute(int[] A, int low, int high) {
    if (low == high) {
        permutations.add(Arrays.copyOf(A, A.length));
        return;
    }
    for (int curr=low; curr <= high; curr++) {
        swap(A, low, curr);
        permute(A, low+1, high);
        swap(A, curr, low);
    }
}
\end{lstlisting}
%Next permutation:
%\begin{lstlisting}[language=java]
%TODO
%\end{lstlisting}{}

\begin{comment}
\subsection{Hash Tables}
\subsubsection{Dynamic - $O(1)$ average, $O(n)$ worst case}
Capacity, load factor, chaining, bloom filter.
\begin{itemize}
    \itemsep0em
    \item Capacity - initial size of primary array.
    \item Load factor - what percentage fill to resize.
    \item Separate chaining - use list for collision.
    \item Linear, quadratic chaining - use $+k$, $_k^2$.
    \item Cuckoo - kick out element.
    \item Bloom filter - use multiple hash to create fingerprint for keys, faster retrieval if not exist.
\end{itemize}
\begin{lstlisting}[language=java]
    class Dictionary {
        List<List<Pair>> d = new ArrayList<>();
        int capacity;
        int hash(int key) { return Math.abs(key) % capacity; }
        Dictionary(int capacity) {
            this.capacity = capacity;
            for (int i=0; i < capacity; i++) d.add(new ArrayList<Pair>());
        }
        void put(int key, String value) {
            for (Pair p : d.get(hash(key))) {
                if (p.key == key) {
                    p.value = value;
                    return;
                }
            }
            d.get(hash(key)).add(new Pair(key, value));
        }
        String get(int key) {
            for (Pair p : d.get(hash(key))) {
                if (p.key == key)
                    return p.value;
            }
            return null;
        }
    }
\end{lstlisting}

\subsubsection{Static}
If you know the keys ahead of time, use primary hash function to find out where they go to, then build secondary perfect hash functions. This will be $O(1)$ worst case and $O(n)$ space, since you know ahead of time how large to make your secondary arrays.
\end{comment}

\subsubsection{Disjoint set}
\begin{lstlisting}[language=java]
class UnionFind {
    int[] nodes;
    public UnionFind(int n) {
        nodes = new int[n];
        for (int i = 0; i < n; i++) nodes[i] = i;
    }
    public int find(int x) {
        if (x != nodes[x]) nodes[x] = find(nodes[x]);
        return nodes[x];
    }
    public void union(int x, int y) { nodes[find(x)] = y; }
    // Can improve w/ union by rank.
}
\end{lstlisting}
\begin{comment}
int parentX = find(x);
        int parentY = find(y);
        // Union by rank
        if (rank[parentX] > rank[parentY]) {
            nodes[parentY] = parentX;
        } else if (rank[parentY] > rank[parentX]) {
            nodes[parentX] = parentY;
        } else {
            nodes[parentX] = parentY;
            rank[parentY]++;
        }
\end{comment}{}


\subsection{Trees and Graphs}
\subsubsection{Binary tree}
\begin{lstlisting}[language=java]
class Node {
    int val; Node left, right; //Map<Character, Node> trieMapping;
}
\end{lstlisting}
\begin{comment}
\begin{lstlisting}[language=java]
BinaryNode(int val) { this.val = val; }
static void traverse(Node root) {
    //Preorder
    if (root.left != null) traverse(root.left);
    // Inorder
    if (root.right != null) traverse(root.right);
    // Postorder
}
\end{lstlisting}

\subsubsection{N-nary tree}
Use array instead of left and right from above, or use an adjacency list graph.
\end{comment}


%\subsubsection{Balanced BST}
%Find the place to insert, then insert as a leaf, as black if RB. Then, rotate up to fix AVL properties (height of subsequent nodes differ by at most 1) or to fix RB properties (black root, no subsequent red nodes, all paths contain same number of black nodes). Rotations: single rotate pushes the largest element down so that the center item is promoted, and double rotate first rotates the bottom two to get the center node in the middle, and then single rotates.
\begin{comment}
\subsubsection{Graph using objects}
%Annoying, not that interesting to implement.
\subsubsection{Graph using adjacency list}
\begin{lstlisting}[language=java]
class Vertex {
    String key, value;
    Vertex(String key, String value) { this.key = key; this.value = value; }
    @Override
    public boolean equals(Object o) { return key.equals(((Vertex) o).key); }
}
class Graph {
    Map<Vertex, List<Vertex>> g;
    Graph() { g = new HashMap<Vertex, List<Vertex>>(); }
    Vertex vertex(String key, String val) {
        Vertex v = new Vertex(key, val);
        if (g.containsKey(v)) return null;
        g.put(v, new ArrayList<>());
        return v;
    }
    void edge(Vertex v1, Vertex v2) {
        if (g.get(v1.contains(v2)) return;
        g.get(v1).add(v2);
    }
    
}
\end{lstlisting}
\subsubsection{Graph using matrix}
\begin{lstlisting}[language=java]
class Graph {
    List<List<int>> g;
    List<String> nodes;
    Graph(List<String> nodes) {
        this.nodes = nodes;
        g = new ArrayList(nodes.size());
        for (int i = 0; i < nodes.size(); i++) {
            g.add(new ArrayList<List<String>>());
            for (int j = 0; j < nodes.size(); j++)
                g.get(i).add(-1);
        }
    }
    void edge(String v1, String v2, int val) {
        g.get(nodes.indexOf(v1)).set(nodes.indxOf(v2), val);
    }
}
\end{lstlisting}
\end{comment}
\subsubsection{BFS, DFS, Topological Sort}
\begin{itemize}
    \item BFS uses queue, while DFS uses stack. Function calls are on a stack, so no other data structures needed.
    \item For topological sorting, intuition is that it is ordered by decreasing finishing time for DFS, so add a node that connects to all nodes without incoming edges. Then, run dfs on it, with tail call to add it to array.
\end{itemize}
\begin{lstlisting}[language=java]
// Assume vertex passed in is already marked as visited.
// Assume that the node passed in can reach every node.
// If not, keep searching unvisited nodes.
Deque<Vertex> dq = new ArrayDeque<>();
List<Vertex> sorted = new ArrayLost<>();
void bfs/dfs(Vertex v) {
    for (Vertex dst : g.get(v)) {
        if (!dst.val.equals("visited") {
            dq.offerLast(dst);
            dst.val = "visited";
        }
    }
    if (dq.size() != 0)
        bfs(dq.pollFirst()); // dfs(dq.pollLast());
}
void dfs(Vertex v) {
    // Process node.
    g.get(v).stream()
        .filter(x -> !x.val.equals("visited"))
        .map(x -> { x.val = "visited"; return x; }
        .forEach(x -> dfs(x));
    //topoSort.add(v);
}
\end{lstlisting}
\subsection{Strongly Connected Components}
Conceptually, two DFS modified to highlight edges that find decreasing last visit time between the nodes, one on the graph and one on the transpose; if highlighted in both runs, SCC. Implementation: DFS with tail call push onto stack (like topological sort, ordering by decreasing last visit time). Reverse directions of edges on graph. While stack is not empty, pop nodes and DFS on unvisited ones to get a single SCC. 
\subsection{Minimum Spanning Tree}
\begin{itemize}
    \itemsep0em
    \item Kruskal's: Store edges weights in a minheap (like sorting the edges). Create disjoint set for nodes. For each minimum edge, if the two vertices are in different trees, union the nodes and add the edge.
    \item Prim's: Create disjoint set for nodes. While there are isolated nodes, iterate through set to get the minimum edge to something we have not visited yet.
\end{itemize}
\subsection{Paths}
\begin{itemize}
    \itemsep0em
    \item Djikstra
    \item Bellman Ford/Floyd Warshall
    \item Matrix multiplication
\end{itemize}
%TODO

%\subsection{Recursion, Divide and Conquer}
%TODO

\begin{comment}
\subsection{System Design, Testing and edge cases}
\begin{itemize}
    \itemsep0em
    \item Breaking down a problem into subproblems
    \item Clarify types of valid/invalid input, input size
    \item Engineering tradeoffs, e.g. time and memory, complexity of solution might be harder to modify
    \item CAP theorem - in distributed systems, the tradeoff between:
    \begin{itemize}
        \item consistency - most recent data or error
        \item availability - no errors, but not necessarily more recent data
        \item partition tolerance - ability to handle data drop. If you cancel the operation, then you are not available but are consistent. On the other hand, if you proceed with the operation, you are available but not consistent.
    \end{itemize}
    \item ACID - database principle, more consistent
    \begin{itemize}
        \item atomicity - all or nothing
        \item consistency - only valid data written
        \item isolation - transactions are independent
        \item durability - no data lost
    \end{itemize}
    \item BASE - database principle, more available
    \begin{itemize}
        \item Basically available
        \item Soft-state
        \item Eventually consistent
    \end{itemize}
    \item DNS for load balancing, CDN caching
    \item Consistent hashing - servers can go in/out of a system, and then you need to move a bunch of data around because your hash based on the number of servers is incorrect. Consistent hashing fixes this, so if a server goes down, only a small amount of entries will need to be moved.
    \item CRUD - create, retrieve, update, delete
\end{itemize}
    \end{comment}

\subsection{Mathematics}
\subsubsection{Counting}
\begin{itemize}
    \itemsep0em
    \item Ordered list without replacement - $\frac{n!}{(n-k)!}$
    \item Ordered list with replacement - $n^k$
    \item Subset without replacement - $\binom{n}{k} = \frac{n!}{k!(n-k)!}$
\end{itemize}
%\subsubsection{Probability}
%TODO

\subsection{Coding}
\subsubsection{OOP}
\begin{itemize}
    \itemsep0em
    %\item Extending a base class or abstract class, or implementing an interface.
    \item Encapsulation\\
    \begin{tabular}{|c|c|c|c|c|c|}
    \hline
         & class & package & subclass & subclass & world \\
         & & & same pkg & diff pkg &\\
         \hline
        public & Y & Y & Y & Y & Y\\
        protected & Y & Y & Y & Y & \\
        none (pkg-private) & Y & Y & Y & & \\
        private & Y & & & &\\
        \hline
    \end{tabular}
    \item Polymorphism - method overloading, method overriding
\end{itemize}

\begin{comment}
\subsubsection{Machine Learning Concepts}
Bias Variance tradeoff, error decomposition\\
Accuracy, ROC Curve, Precision/Recall/F1 score, Type I/II Error\\
Baye's theorem, MLE, MAP\\
Overfitting, regularization, cross validation\\
Imbalance\\
Ensemble methods, Boosting, Random Forests\\
Kernel trick\\
Models:
\begin{itemize}
    \itemsep0em
    \item Linear Regression, SVM
    \item Logistic Regression Binary/Softmax Multiclass classifier
    \item Decision Trees
    \item Naive Bayes Classifier
    \item Gaussian Mixture, EM method
    \item Neural network
\end{itemize}
TODO
\end{comment}

%\subsection{Development Practices and Open-Ended Discussion}
%\begin{itemize}
%    \itemsep0em
%    \item Validating designs
%    \item Testing whiteboard code
%    \item Preventing bugs
%    \item Code maintainability and readability
%    \item Refactor/review sample code
%    \item Biggest challenge faced
%    \item Best/worst design seen
%    \item Performance analysis and optimization
%    \item Testing and ideas for improving existing products
%\end{itemize}

\end{document}
