
\subsection{Sorting and searching}

\subsubsection{Binary search - $O(\log n)$}
\textbf{Recursive}:
\begin{lstlisting}[language=java]
int search(int[] A, int key, int low, int high) {
    if (high < low) {
        // return low; // ceiling
        // return high; // floor
        return -1;
    }
    int mid = low + (high-low)/2;
    if (A[mid] == key) return mid;
    else if (A[mid] > key) return search(A, key, low, mid-1);
    else return search(A, key, mid+1, high);
}
\end{lstlisting}
\textbf{Iterative}:
\begin{lstlisting}[language=java]
int search(int[] A, int key) {
    int low = 0; int high = A.length-1;
    while (low <= high) {
        int mid = low + (high - low) / 2;
        if (A[mid] == key) return mid;
        if (key < A[mid]) high = mid-1;
        if (key > A[mid]) low = mid+1;
    }
    return -1;
}
\end{lstlisting}{}

\subsubsection{Merge sort - $O(n\log n)$}
Partition into smaller and smaller arrays, then merge to sort.
\begin{lstlisting}[language=java]
void sort(int[] A) {
    if (A.length <= 1) return;
    
    int mid = A.length/2;
    int[] left = new int[mid];
    int[] right = new int[A.length-mid];
    
    System.arraycopy(A, 0, left, 0, mid);
    System.arraycopy(A, mid, right, 0, A.length-mid);
    sort(left); sort(right); merge(A, left, right);
}
void merge(int[] A, int[] left, int[] right) {
    int l, r, i; l = r = i = 0;
    while (l < left.length || r < right.length) {
        if (l < left.length && r < r.length)
            A[i++] = compare(left[l], right[r]) ? right[r++] : left[l++];
        else if (l < left.length)
            A[i++] = left[l++];
        else if (r < right.length)
            A[i++] = right[r++];
    }    
}
\end{lstlisting}

\subsubsection{Quick sort - $O(n^2)$ worst case, $O(n\log n)$ average case}
%Partition into higher and lower group, then recursively sort.
\begin{lstlisting}[language=java]
void sort(int[] A, int low, int high) {
    if (high < low) return;
    int pivot = partition(A, low, high);
    sort(A, low, pivot-1); sort(A, pivot+1, high);
}
int partition(int[] A, int low, int high) {
    int insPt = low-1;
    for (int curr = low; curr < high; curr++)
        if (A[curr] <= A[high]) swap(A, ++insPt, curr);
    swap(A, ++insPt, high);
    return insPt;
}
\end{lstlisting}

\begin{comment}
\subsubsection{Count sort - $O(n+r)$}
Count frequencies of each element, then unpack.
\begin{lstlisting}[language=java]
void sort(int[] A) {
    if (A.length == 0) return;
    int min = Arrays.stream(A).min().getAsInt();
    int max = Arrays.stream(A).max().getAsInt();
    int[] counts = new int[max-min+1];
    for (int i : A) counts[i-min]++;
    
    int j = 0;
    for (int i=0; i < A.length; i++) {
        while (counts[j] == 0) j++;
        A[i] = min + j;
        counts[j]--;
    }
}
\end{lstlisting}
\end{comment}


\subsubsection{Radix sort - $O(nk/d)$}
\begin{lstlisting}[language=java]
void sort(int[] A) {                                  
    for (int i = 0; i < MAX_LEN; i++) {
        List<List<Integer>> B = new ArrayList<>(BUCKETS);                  
        for (int b = 0; b < BUCKETS; b++) B.add(new ArrayList<>());
        for (int a : A) B.get(placeValue(a, i)).add(a);
        for (int b = 0, unpacker = 0; b < BUCKETS; k++)
            for (int a : B.get(b))
                A[unpacker++] = a;                           
    }
}
\end{lstlisting}

\subsubsection{Median/order stats}
\textbf{Median of medians - $O(n)$ worst and best case}\\
\textbf{Quick select - $O(n^2)$ worst case but $O(n)$ average}\\
Partition, then recurse based on the size of partition created by pivot.
\begin{lstlisting}[language=java]
public static int search(int[] A, int k, int low, int high) {
    if (high == low) return A[low];
    int pivot = partition(A, low, high);
    if (k < pivot) return search(A, k, low, pivot-1);
    else if (k > pivot) return search(A, k, pivot+1, high);
    else return A[pivot];
}
\end{lstlisting}







\begin{comment}
\subsubsection{Insertion sort - $O(n^2)$}
Preserves loop invariant by inserting next element right of cursor into invariant.
\begin{itemize}
    \itemsep0em 
    \item Adaptive - good for mostly sorted data sets.
    \item Stable - preserves order of equal items.
    \item In-place - no auxillary data structures.
    \item Online - does not need whole input to start sorting.
    \item Insertion finding step can be sped up with binary search.
\end{itemize}
\begin{lstlisting}[language=java]
void sort(int[] A) {
    for (int cursor=1; cursor < A.length; cursor++) {
        int cursorVal = A[cursor];
        int left = cursor - 1;
        while (left >= 0 && A[left] > cursorVal) {
            A[left+1] = A[left];
            left--;
        }
        A[left+1] = cursorVal;
    }
}
\end{lstlisting}

\subsubsection{Selection sort - $O(n^2)$}
Preserves loop invariant by putting the next largest element right of cursor as next element of invariant.   
\begin{lstlisting}[language=java]
sort(int[] A) {
    for (int cursor=0; cursor < A.length-1; cursor++) {
        int minIndex = cursor;
        for (int right = cursor+1; right < A.length; right++) {
            if (compare(A[minIndex], A[right])
                minIndex = right;
        }
        if (minIndex != cursor)
            swap(A, cursor, minIndex);
    }
}
\end{lstlisting}

\subsubsection{Bubble sort}
Not very good or interesting...
\end{comment}